\documentclass[12pt,twoside]{article}
\usepackage[dvipsnames]{xcolor}
\usepackage{tikz,graphicx,amsmath,amsfonts,amscd,amssymb,bm,cite,epsfig,epsf,url}
\usepackage[hang,flushmargin]{footmisc}
\usepackage[colorlinks=true,urlcolor=blue,citecolor=blue]{hyperref}
\usepackage{amsthm,multirow,wasysym,appendix}
\usepackage{array,subcaption} 
% \usepackage[small,bf]{caption}
\usepackage{bbm}
\usepackage{pgfplots}
\usetikzlibrary{spy}
\usepgfplotslibrary{external}
\usepgfplotslibrary{fillbetween}
\usetikzlibrary{arrows,automata}
\usepackage{thmtools}
\usepackage{blkarray} 
\usepackage{textcomp}
\usepackage[left=0.8in,right=1.0in,top=1.0in,bottom=1.0in]{geometry}

\usepackage{times}
\usepackage{amsfonts}
\usepackage{amsmath}
\usepackage{latexsym}
\usepackage{color}
\usepackage{graphics}
\usepackage{enumerate}
\usepackage{amstext}
\usepackage{blkarray}
\usepackage{url}
\usepackage{epsfig}
\usepackage{bm}
\usepackage{hyperref}
\hypersetup{
    colorlinks=true,
    linkcolor=blue,
    filecolor=magenta,      
    urlcolor=blue,
}
\usepackage{textcomp}
\usepackage[left=0.8in,right=1.0in,top=1.0in,bottom=1.0in]{geometry}
\usepackage{mathtools}
\usepackage{minted}

\input{macros}
%\input{Symbols}

\begin{document}
$$
	E_{\text{ion}} = \frac{61.56}{Z} \cdot \log_{10} \bigg( \frac{C_{\text{out}}} {C_{\text{in}}} \bigg)
$$
where 
\begin{enumerate}[] 
\item $E_{\text{ion}} =$ equilibrium potential or Ernst potential for a particular ion, in mV
\item $C_{\text{in}} =$ intracellular concentration of the ion, in mM/L
\item $C_{\text{out}} =$ extracellular concentration of the ion, in mM/L
\item $Z$ = the valence of the ion
\end{enumerate}
 Using the concentration gradients from the table in the question, the Nersnt potentials for Na$^+$, K$^+$ and Cl$^-$  are:
 \begin{align*}
 	E_{\text{Na}^+}	&= \frac{61.56}{+1} \log_{10} \bigg( \frac{127} {10} \bigg) \approx 68 \text{mV} \\
 	E_{\text{K}^+}	&= \frac{61.56}{+1} \log_{10} \bigg( \frac{4} {120} \bigg) \approx -90 \text{mV} \\
 	E_{\text{Cl}^-}	&= \frac{61.56}{-1} \log_{10} \bigg( \frac{80} {2} \bigg) \approx -98 \text{mV} \\
 \end{align*}
% The Cl$^-$ concentration is reversed because Cl$^-$ is an anion and its movement has the opposite effect on the membrane potential.

First knowing that the plasma membrane behaves electrically like a capacitor of capacitance C$_\text{a}$ F/m$^2$, the charge separation across the
membrane is:
 \begin{align*}
 	Q &= C_\text{a} \cdot (2 \pi R^2 + 2 \pi R L) \cdot V_{\text{m}}\\
	    &= C_\text{a} \cdot 2 \pi R (R + L)  \cdot V_{\text{m}}
 \end{align*}
 
 The amount of electricity that is carried by 1 mole of electrons is given by Faraday's constant: F (96,480 coulombs/mole).
 The amount of ion X moved to establish $V_{\text{m}}$ is: 
 \begin{align*}
 X_{i \rightarrow o} 	&= \frac{Q}{F} \\
 				&= \frac{C_\text{a} \cdot 2 \pi R (R + L)  \cdot V_{\text{m}}} {F}
 \end{align*}
 With a cylindrical cell of volume $\pi R^2 L$, the initial amount of ion X with an intracellular concentration of $[X]_{i}$ is:
 $$X_{\text{i}} = \pi R^2 L [X]_{i}$$
 Thus the ratio of the amount (in mol) of ion X moved to establish $V_{\text{m}}$ to the initial amount of ion X is:
 $$\frac{X_{i \rightarrow o} } {X_{\text{i}}} = 2 \cdot \frac{C_\text{a}} {F} \frac{R+L} {R \cdot L} \frac{V_{\text{m}}} {[X]_{i}}$$
 
 %  P_{\text{Na}} \cdot [Na]_o + P_K [K]_o + P_{\text{Cl}} [Cl]_i
 GHK equation: 
 $$
 	V_{\text{GHK, Na, K, Cl}} = \frac{R \cdot T} {F} \cdot \ln \bigg( \frac{P_{\text{Na}} \cdot [\text{Na}]_o  + P_K [K]_o + P_{\text{Cl}} [Cl]_i} 
	{P_{\text{Na}} \cdot [\text{Na}]_i  + P_K [K]_i + P_{\text{Cl}} [Cl]_o} \bigg)
 $$
 
 $$
 	V_{\cl}	= -  \frac{R \cdot T} {F} \cdot \ln \bigg(   \frac{ [\cl]_o} {[\cl]_i}  \bigg)
 $$
 \begin{align*}
 	E_m	&= \frac{g_{\text{Na}^+}} {g_{\text{Na}^+} + g_{\text{K}^+}} E_{\text{Na}^+} +  \frac{g_{\text{K}^+}} {g_{\text{Na}^+} + g_{\text{K}^+}} E_{\text{K}^+} \\
	\text{let } r &=  \frac{g_{\text{Na}^+}} {g_{\text{K}^+}} \\
	E_m	&= \frac{r} {1+r} E_{\text{Na}^+} + \frac{1}{1+r} E_{\text{K}^+} \\
	& \text{multiplying on both sides by} (1+r) \\
	(1+r) E_m	&=   r E_{\text{Na}^+} + E_{\text{K}^+} \\
	r &= \frac{ E_{\text{K}^+}  -  E_m} {  E_m - E_{\text{Na}^+} }
 \end{align*}
Using the concentrations of Na$^+$ and K$^+$ and their computed Nernst potentials as given in slide 8 of video 7, the ratio is:
\begin{align*}
	r &= \frac{g_{\text{Na}^+}} {g_{\text{K}^+}}\\
	  &= \frac{ E_{\text{K}^+}  -  E_m} {  E_m - E_{\text{Na}^+} } \\
	  &= \frac{-94 - (-86)} {-86-61} \\
	  &= \frac{8}{147} \\
	  &= 0.054 ~ \text{or } 5.4\%
 \end{align*}

 
 Given:
 $$
 	\frac{X^2} {(U-X) (T-X)} = \frac{A^2} { (U-A) (R-A)}
 $$
 Let $\frac{1}{\alpha} =  \frac{A^2} { (U-A) (R-A)}$ then 
 \begin{align*}
 	\alpha X^2 	&=	(U-X) (T-X) \\
	(1-\alpha) X^2 - (U+T) X + UT &= 0 \\
  \end{align*}
  
Solving the quadratic assuming validity of the discriminant for real values:
$$
	X_{0,1} = \frac { U + T \pm \sqrt{ (U - T) ^2 + 4 \alpha U T}} {2 ( 1 - \alpha)}
$$

If we set $c = \frac{1}{\alpha}$ we obtain the reach curve white paper solution:
$$
	X_{0,1} = \frac { c (U + T) \pm \sqrt{c}  \sqrt{ c (T - U) ^2 + 4 U T}} {2 ( -1 + c)}
$$

\end{document}